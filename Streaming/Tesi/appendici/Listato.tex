\chapter{Listato} \label{chap:Listato}
In questa appendice sono riportate alcune parti del codice C++ a cui si fa riferimento nel testo. Per ogni listato è riportato il file da cui è stato estratto.

\section{Modifiche classi SDL}
Di seguito le modifiche che sono state apportate alle classi originali del MAME nel main e per recuperare il rendering e il missaggio audio.

\subsection{Main}
File \verb|osd/sdl/sdlmain.cpp|

\begin{lstlisting}
int main(int argc, char** argv)
{
	auto r = 0;	
	streaming_server::instance().activate(argc, argv);

	if (streaming_server::instance().is_active())
	{
		streaming_server::instance().on_accept = [&](auto parameters)
		{
			streaming_server::run_new_process(argc, argv);

			r = main_sdl(argc, argv, parameters["game"]);
		};

		streaming_server::instance().start(8888);
	}
	else
		r = main_sdl(argc, argv, "");	

	return r;
}
\end{lstlisting}


\subsection{Missaggio audio} \label{lst:sdl_sound}
File \verb|osd/modules/sound/sdl_sound.cpp|

\begin{lstlisting}
void sound_sdl::sdl_callback(
	void* userdata, Uint8* stream, int len)
{
	//... CODICE NON MODIFICATO ...

	if (streaming_server::instance().is_active())
	{
		streaming_server::instance().send_audio_interval(
			stream, len, this_->sdl_xfer_samples);

		memset(stream, 0, len); //silence local outputs
	}

	//... CODICE NON MODIFICATO ...
}
\end{lstlisting}

\subsection{Rendering}
Files \verb|osd/modules/render/draw13.cpp| e \verb|osd/modules/render/draw13.h|

\begin{lstlisting}[label={lst:draw13}]
void renderer_sdl2::free_streaming_render() const
{
	if (m_sdl_buffer_bytes != nullptr)
	{
		SDL_RWclose(m_sdl_buffer);
		SDL_FreeSurface(m_sdl_surface);
		SDL_DestroyRenderer(m_sdl_renderer);

		delete[] m_sdl_buffer_bytes;
	}
}

void renderer_sdl2::init_streaming_render(
	const int w, const int h, const int fps)
{
	free_streaming_render();

	m_sdl_buffer_bytes_length = w * h * 4;
	m_sdl_buffer_bytes = new char[m_sdl_buffer_bytes_length];

	m_sdl_surface = SDL_CreateRGBSurfaceWithFormat(
		0, w, h, 32, SDL_PIXELFORMAT_RGBA32);

	//Crea il rendering su superfice al posto di quello su finestra
	m_sdl_renderer = SDL_CreateSoftwareRenderer(
		m_sdl_surface);

	m_sdl_buffer = SDL_RWFromMem(
		m_sdl_buffer_bytes, m_sdl_buffer_bytes_length);

	if (streaming_server::instance().is_active())
		streaming_server::instance().init_encoding(w, h, fps);
}

int renderer_sdl2::create()
{
	//... CODICE NON MODIFICATO ...

	if (streaming_server::instance().is_active())
	{
		const auto nd = win->get_size();
		init_streaming_render(
			nd.width(), nd.height(), win->m_win_config.refresh);
	}
	else
	{
		if (video_config.waitvsync)
			m_sdl_renderer = SDL_CreateRenderer(
				dynamic_pointer_cast<sdl_window_info>(
					win)->platform_window(), 
					-1, SDL_RENDERER_PRESENTVSYNC | 
					SDL_RENDERER_ACCELERATED);
		else
			m_sdl_renderer = SDL_CreateRenderer(
				dynamic_pointer_cast<sdl_window_info>(
					win)->platform_window(), 
					-1, SDL_RENDERER_ACCELERATED);
	}

	//... CODICE NON MODIFICATO ...
}

int renderer_sdl2::draw(int update)
{
	//... CODICE NON MODIFICATO ...
	SDL_RenderPresent(m_sdl_renderer);

	if (streaming_server::instance().is_active())
	{
		streaming_server::instance().send_video_frame(
			static_cast<uint8_t*>(m_sdl_surface->pixels));
		SDL_RWseek(m_sdl_buffer, 0, RW_SEEK_SET);
	}
	return 0;
}
\end{lstlisting}

\subsection{Gestione input}
Files \verb|osd/modules/input/input_sdl.cpp|

\begin{lstlisting}
sdl_keyboard_device(
		running_machine &machine, const char *name,
		const char *id, input_module &module): 
	sdl_device(machine, name, id, DEVICE_CLASS_KEYBOARD, module),
	keyboard({{0}})
{
	streaming_server::instance().set_keyboard(this);
}
\end{lstlisting}

\section{Moduli nuovi}
Di seguito alcune funzioni nei due moduli che sono stati creati: codifica e server per la comunicazione WebSocket.


\subsection{Codifica}
File \verb|lib/util/encoding/encode_to_movie.hpp|

\begin{lstlisting}
static int write_packet(
	void* opaque, uint8_t* buf, int buf_size)
{
	auto* const this_ = 
		static_cast<encode_to_movie*>(opaque);

	this_->socket->write(
		reinterpret_cast<const char*>(buf), buf_size);

	return 1; //1 element wrote
}

//Add video frame
void add_frame(const uint8_t* pixels)
{
	if (!header)
		write_header();

	avpicture_fill(
		reinterpret_cast<AVPicture*>(rgb_frame),
		pixels, PIXEL_FORMAT_IN, width, height);
	
	//RGB to YUV
	sws_scale(
		encoder_context->video_converter_context,
		rgb_frame->data, rgb_frame->linesize, 0, height,
		yuv_frame->data, yuv_frame->linesize
	);

	av_init_packet(&video_packet);
	video_packet.data = nullptr;
	video_packet.size = 0;

	avcodec_send_frame(
		encoder_context->video_codec_context, yuv_frame);

	const auto got_packet_ptr = 
		avcodec_receive_packet(
			encoder_context->video_codec_context, 
			&video_packet) == 0;

	if (got_packet_ptr)
	{
		const auto ret = 
			av_interleaved_write_frame(
				encoder_context->muxer_context, 
				&video_packet);

		if (ret == 0)
			send_it();
		else
			error("Error while writing video frame", ret);
	}

	av_packet_unref(&video_packet);
}

//Add audio instant
void add_instant(
	const uint8_t* audio_stream, 
	const int audio_stream_size, 
	const int audio_stream_num_samples)
{
	if (!header)
		write_header();

	wav_frame->nb_samples = audio_stream_num_samples;

	auto ret = avcodec_fill_audio_frame(
		wav_frame,
		AUDIO_CHANNELS_IN,
		AUDIO_SAMPLE_FORMAT_IN,
		audio_stream,
		audio_stream_size,
		1); //no-alignment
	if (ret < 0)
		die("Cannot fill audio frame", ret);

	if (convertedData == nullptr)
	{
		ret = av_samples_alloc(
			&convertedData,
			nullptr,
			AUDIO_CHANNELS_OUT,
			aac_frame->nb_samples,
			AUDIO_SAMPLE_FORMAT_OUT,
			0);			
		if (ret < 0)
			die("Could not allocate samples", ret);
	}

	auto outSamples = swr_convert(
		encoder_context->audio_converter_context,
		//output
		nullptr, 0,
		//input
		const_cast<const uint8_t**>(
			wav_frame->data), wav_frame->nb_samples
	);

	if (outSamples < 0)
		die("Could not convert", outSamples);

	for (;;)
	{
		outSamples = swr_get_out_samples(
			encoder_context->audio_converter_context, 0);

		if (outSamples < 
			encoder_context->audio_codec_context->frame_size * 
			encoder_context->audio_codec_context->channels)
			break;

		swr_convert(
			encoder_context->audio_converter_context,
			//output
			&convertedData, aac_frame->nb_samples,
			//input
			nullptr, 0);

		const auto buffer_size = av_samples_get_buffer_size(
			nullptr,
			encoder_context->audio_codec_context->channels,
			aac_frame->nb_samples,
			encoder_context->audio_codec_context->sample_fmt,
			0);

		if (buffer_size < 0)
			die("Invalid buffer size", buffer_size);

		ret = avcodec_fill_audio_frame(
			aac_frame,
			encoder_context->audio_codec_context->channels,
			encoder_context->audio_codec_context->sample_fmt,
			convertedData,
			buffer_size,
			0);

		if (ret < 0)
			die("Could not fill frame", ret);

		av_init_packet(&audio_packet);
		audio_packet.data = nullptr;
		audio_packet.size = 0;

		int got_packet_ptr;

		ret = avcodec_encode_audio2(
			encoder_context->audio_codec_context, 
			&audio_packet, aac_frame, 
			&got_packet_ptr);

		if (ret < 0)
			die("Error encoding audio frame", ret);

		if (got_packet_ptr)
		{
			audio_packet.stream_index = 1;

			ret = av_interleaved_write_frame(
				encoder_context->muxer_context, &audio_packet);

			if (ret == 0)
				send_it();
			else
				error("Error while writing audio frame", ret);
		}

		av_packet_unref(&audio_packet);
	}
}
\end{lstlisting}


\newpage
\subsection{Server}
File \verb|lib/util/streaming_server.hpp|

\begin{lstlisting}
//Invio input utente a SDL
void generate_key_event(
	const char* key, const string& down) const
{
	SDL_Event e;
	e.type = (down == "D" ? SDL_KEYDOWN : SDL_KEYUP);

	e.key.keysym.scancode = 
		SDL_GetScancodeFromName(key);
	
	e.key.keysym.sym = 
		SDL_GetKeyFromScancode(e.key.keysym.scancode);

	keyboard->queue_events(&e, 1);
}

//Singleton
static streaming_server& instance()
{
	static streaming_server instance;
	return instance;
}

void start(const unsigned short port)
{
	server = make_unique<ws_server>();
	server->config.client_mode = true;
	server->config.port = port;
	server->config.timeout_request = 0; //no timeout

	auto& endpoint = server->m_endpoint["/?"];

	endpoint.on_open = [this](auto connection)
	{
		cout
			<< "-Opened connection from "
			<< connection->remote_endpoint_address << ":"
			<< connection->remote_endpoint_port	<< endl;

		game_thread = make_unique<thread>(
			on_accept, connection->parameters);

		game_start_time = chrono::system_clock::now();

		const auto game = connection->parameters["game"];

		cout
			<< "Starting game: " << game
			<< endl;
	};

	endpoint.on_message = [this](auto connection, auto message)
	{
		const auto msg = message->string();
		const auto values = ws_server::split(msg, ":");

		if (values[0] == "ping")
			process_pausing_mechanism();
		else if (values[0] == "key")
			process_key(values);
	};
	
	endpoint.on_error = [](auto connection, auto code)
	{
		cout
			<< "-Error on connection from "
			<< connection->remote_endpoint_address << ":"
			<< connection->remote_endpoint_port	<< endl
			<< ": " << code.message() << endl;
	};

	endpoint.on_close = [this](auto connection, 
		auto status, auto reason)
	{
		const auto game = connection->parameters["game"];
		const auto game_end_time = chrono::system_clock::now();

		const auto game_total_minute_played = 
			chrono::duration_cast<chrono::minutes>(
				game_end_time - game_start_time);		

		cout
			<< "-Closed connection from "
			<< connection->remote_endpoint_address << ":"
			<< connection->remote_endpoint_port	<< endl
			<< ": " << reason << endl;

		cout
			<< game
			<< " played for: " 
			<< game_total_minute_played.count()
			<< "min." << endl;

		machine->schedule_exit();
	};

	cout
		<< "Game streaming server listening on " 
		<< port	<< endl;

	server->start();

	if (game_thread->joinable())
		game_thread->join();
}
\end{lstlisting}