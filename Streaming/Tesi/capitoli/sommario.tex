%license:BSD-3-Clause
%copyright-holders:Michele Maione
%============================================================
%
%	Piattaforma di cloud gaming per giochi arcade
%
%============================================================

\chapter*{Sommario}
\addcontentsline{toc}{chapter}{Sommario}

%Il sommario deve contenere 3 o 4 frasi tratte dall’introduzione di cui la prima inquadra l’area dove si svolge il lavoro (eventualmente la seconda inquadra la sottoarea più specifica del lavoro), la seconda o la terza frase dovrebbe iniziare con le parole “Lo scopo della tesi è . . . ” e infine la terza o quarta frase riassume brevemente l’attività svolta, i risultati ottenuti ed eventuali valutazioni di questi.
Negli ultimi anni sono apparse molte piattaforme che sfruttano il paradigma del cloud computing per offrire servizi accessibili su richiesta e da remoto per archiviare file, utilizzare le suite per l'ufficio, vedere film e serie TV, ascoltare musica e a partire dal 2011 anche giocare.

Il cloud gaming è un servizio che unisce il cloud computing e il live streaming per rendere possibile giocare in remoto senza scaricare o installare il gioco sul device dell'utente, in pratica consente di archiviare ed eseguire i videogiochi su un server remoto e trasmettere l'output audio-video all'utente sul proprio dispositivo.

Per far conoscere alle nuove generazioni i videogiochi che hanno fatto la storia e dare la possibilità di poter giocare ancora a macchine che ormai hanno cessato di funzionare per motivi di obsolescenza, sfruttando due tecnologie entrate a far parte della quotidianità, il live streaming e il cloud computing, in questo lavoro si propone la creazione di una piattaforma di cloud gaming. La piattaforma permetterà lo streaming audio-video, direttamente e su richiesta, dei videogiochi da un server remoto ad un client (computer, console e telefono). Il gioco è archiviato, eseguito e renderizzato su un server remoto; l'input (tastiera e gamepad) viene inviato dal client al server e lì processato. Il cloud gaming permette di iniziare a giocare immediatamente poiché il gioco è già installato sul server offrendo agli utenti un rapido accesso indipendentemente dal sistema operativo e dalle capacità hardware del client utilizzato. Infine la piattaforma, indirettamente, garantisce la gestione dei diritti digitali (DRM) per gli editori. Per questo progetto verrà ampliato il software MAME (rilasciato sotto licenza GNU-GPL) che è in grado di emulare oltre 7.000 giochi arcade, in modo che possa fungere da server di cloud gaming e comunicare con un front-end HTML, rimanendo sempre indipendente dal sistema operativo, così da rendere più agevole l’installazione di uno stand per il retro-gaming.