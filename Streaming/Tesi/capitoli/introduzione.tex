%license:BSD-3-Clause
%copyright-holders:Michele Maione
%============================================================
%
%	Piattaforma di cloud gaming per giochi arcade
%
%============================================================

\chapter*{Introduzione}
\addcontentsline{toc}{chapter}{Introduzione}
\markboth{}{INTRODUZIONE}

%L'introduzione deve essere atomica, quindi non deve contenere nè sottosezioni nè paragrafi nè altro. Il titolo, il sommario e l'introduzione devono sembrare delle scatole cinesi, nel senso che lette in quest'ordine devono progressivamente svelare informazioni sul contenuto per incatenare l'attenzione del lettore e indurlo a leggere l'opera fino in fondo. L'introduzione deve essere tripartita, non graficamente ma logicamente:
%Il capitolo che apre questa tesi



%Inquadramento generale
%La prima parte contiene una frase che spiega l'area generale dove si svolge il lavoro; una che spiega la sottoarea più specifica dove si svolge il lavoro e la terza, che dovrebbe cominciare con le seguenti parole "lo scopo della tesi è \dots", illustra l'obbiettivo del lavoro. Poi vi devono essere una o due frasi che contengano una breve spiegazione di cosa e come è stato fatto, delle attività sperimentali, dei risultati ottenuti con una valutazione e degli sviluppi futuri. La prima parte deve essere circa una facciata e mezza o due

Nel 1972 la società Atari pubblicava il primo videogioco della storia, Pong, vendendo 19.000 cabinati e presto molte altre società seguirono l'esempio. Alla fine del decennio iniziò l'epoca d'oro dei videogiochi arcade e la nascita delle console \parencite{High_Score}. I videogiochi uniscono narrativa, animazione e musica all'interattività, ed è grazie a quest'ultima che riescono ad esercitare un potenziale d'immersione e attrazione che gli altri media non hanno, tanto da diventare un fenomeno culturale di massa con centinaia di milioni di persone che giocano regolarmente ogni giorno, rendendoli attori dominanti nel settore dell'intrattenimento. L'importanza economica dei videogiochi arcade, negli ultimi vent'anni, è notevolmente diminuita a favore dei videogiochi per personal computer, console e più recentemente per mobile.

Il cloud computing è un paradigma a cui siamo ormai abituati e ci risulterebbe difficile abbandonare servizi come Dropbox, Office 365, Spotify e Netflix per tornare alle loro versioni "precedenti": i rullini fotografici, i documenti aziendali negli archivi, i CD audio ed i DVD a noleggio. Dalla nascita dei primi servizi di cloud computing nel 2006 alcuni oggetti sono stati sostituiti con la loro controparte informatica portando al fallimento di aziende storiche come Blockbuster (nel 2013), Kodak (nel 2012) e Borders\footnote{Borders Group era un rivenditore americano di libri e musica.} (nel 2011) \parencite{I_4_fallimenti_più_clamorosi_del_decennio}. Il cloud computing consiste nella distribuzione on-demand delle risorse IT tramite internet su tre livelli di servizio che sono: l'accesso all'infrastruttura hardware tramite API (IaaS), la piattaforma software inclusa di sistemi di sviluppo (PaaS) e le applicazioni (SaaS). Con il cloud computing la potenza della macchina, sia essa fisica o virtuale, aumenta automaticamente all'esigenza permettendo di gestire i picchi di utilizzo; svincola gli utenti dal dover acquistare, manutenere e gestire fisicamente le infrastrutture IT; fornisce l'accesso alle risorse informatiche da qualsiasi device, da qualsiasi luogo e in modo collaborativo.

Unendo il paradigma del cloud computing con lo streaming nasce un nuovo tipo di servizio dedicato ai videogiochi: il cloud gaming. Questo servizio rende possibile giocare in remoto senza scaricare o installare il gioco sul device dell'utente. Con il cloud gaming i videogiochi sono archiviati ed eseguiti su un server remoto e l'output audio-video trasmesso al dispositivo dell'utente, permettendo di iniziare a giocare immediatamente, indipendentemente dal sistema operativo e dalle capacità hardware del dispositivo utilizzato. Infine, indirettamente, viene garantita la gestione dei diritti digitali (DRM) per gli editori. Il cloud gaming è l'unione di due modelli del cloud computing: il modello SaaS e il modello PaaS sia perché il videogioco viene offerto al giocatore come "applicazione" sia perché allo sviluppatore viene offerto il sistema operativo e i vari SDK, questo paradigma implementato dal cloud gaming è definito "gioco come servizio" (GaaS) che si divide in tre instanze: rendering remoto (RR-GaaS), rendering locale (LR-GaaS), allocazione delle risorse cognitive (CRA-GaaS) \parencite{Cloud_for_Gaming}.

%Breve descrizione del lavoro
%La seconda parte deve essere una esplosione della prima e deve quindi mostrare in maniera più esplicita l'area dove si svolge il lavoro, le fonti bibliografiche più importanti su cui si fonda il lavoro in maniera sintetica (una pagina) evidenziando i lavori in letteratura che presentano attinenza con il lavoro affrontato in modo da mostrare da dove e perchè è sorta la tematica di studio. Poi si mostrano esplicitamente le realizzazioni, le direttive future di ricerca, quali sono i problemi aperti e quali quelli affrontati e si ripete lo scopo della tesi. Questa parte deve essere piena (ma non grondante come la sezione due) di citazioni bibliografiche e deve essere lunga circa 4 facciate.

Lo scopo di questa tesi è creare una piattaforma di cloud gaming per far conoscere alle nuove generazioni i videogiochi che hanno fatto la storia e dare la possibilità di poter giocare ancora a macchine che ormai hanno cessato di funzionare per motivi di obsolescenza. Per far ciò verrà ampliato il software MAME (rilasciato sotto licenza GNU-GPL) che è in grado di emulare oltre 7.000 giochi arcade, in modo che possa fungere da server di cloud gaming e comunicare con un front-end HTML, rimanendo sempre indipendente dal sistema operativo, rendendo più agevole l’installazione di uno stand per il retro-gaming. 

Per la realizzazione del progetto sono state prese in considerazione le analisi fatte sulle piattaforme delle multinazionali come GeForce Now, Stadia e PlayStation Now in \parencite{A_Network_Analysis_on_Cloud_Gaming_Stadia_GeForce_Now_and_PSNow} e \parencite{Network_Analysis_of_the_Sony_Remote_Play_System}, come Amazon Luna in \parencite{Amazon_Luna_WebRTC}, ma anche progetti open source come "Games on Demand" di \parencite{ARealTimeStreamingGamesonDemandSystem} che propone una piattaforma basata sul hooking di funzioni DirectX, codifica in MPEG-2 e trasmissione tramite UDP; "GamingAnywhere" di \parencite{GamingAnywhere} che è un progetto multipiattaforma che trasmette tramite protocollo RTP ed esegue la cattura audio-video utilizzando la libreria SDL tramite polling.

Il sistema proposto è stato progettato con un'ottica incentrata sull'utilizzo in LAN con l'utenza connessa tramite WiFi, ad esempio in stand di retro-gaming ad eventi di informatica e videogiochi, in aziende come servizio di svago per i clienti in sala d'attesa e per i dipendenti durante la pausa, ecc\dots, è importante ricordare che i videogiochi, nonostante siano stati pensati come fonte d'intrattenimento, migliorano diversi tipi di abilità chiave: abilità sociali e intellettuali, riflessi e concentrazione \parencite{Use_of_Cloud_Gaming_in_Education}; per questo motivo la piattaforma può essere installata anche nelle scuole.

Per ampliare il progetto MAME, lato server ho modificato le funzionalità di rendering video e missaggio audio per convogliare il loro output, che viene codificato in MPEG-TS, ad un modulo che esegue lo streaming tramite il protocollo WebSocket ad una pagina HTML. Lato client ho creato un modulo JavaScript per gestire l'input utente e decodificare il filmato MPEG-TS. La piattaforma utilizza un bit-rate tra 0.5 Mbps e 2.2 Mbps ed offre una risoluzione di 480p; per valutarne la latenza è stata testata su rete locale e su rete internet \parencite{Latency_analysis_for_M2M}; mentre la qualità audio-video è stata classificata tramite "peek signal-to-noise ratio" e "structural similarity index method" \parencite{Cloud_Gaming_Architecture_and_Performance}.



%Struttura della tesi
%La terza parte contiene la descrizione della struttura della tesi ed è organizzata nel modo seguente.
%La tesi è strutturata nel modo seguente:
%Nella sezione due si mostra \dots
%Nella sez. tre si illustra \dots
%Nella sez. quattro si descrive \dots
%Nelle conclusioni si riassumono gli scopi, le valutazioni di questi e le prospettive future \dots
%Nell'appendice A si riporta \dots (Dopo ogni sezione o appendice ci vuole un punto).
%I titoli delle sezioni da 2 a M-1 sono indicativi, ma bisogna cercare di mantenere un significato equipollente nel caso si vogliano cambiare. Queste sezioni possono contenere eventuali sottosezioni.

La tesi è strutturata nel modo seguente:
\begin{itemize}
	\item il capitolo 1 fornisce un'introduzione sulla nascita dei videogiochi e dei ricavi globali dell'industria videoludica, dà una definizione di cloud computing e di cloud gaming, fà una panoramica delle piattaforme di gioco che si sono susseguite nel tempo e delle proiezioni di mercato del settore del cloud gaming;
	\item nel capitolo 2 verrà descritto il sistema proposto, il MAME e le sue funzioni di: rendering, missaggio audio e gestione dell'input utente;
	\item nel capitolo 3 verranno descritte le cinque fasi aggiuntive del cloud gaming e la loro implementazione in C++ come nuovi moduli del MAME: la cattura audio-video, la codifica, la trasmissione, la decodifica e la gestione dell'input utente;
	\item il capitolo 4 analizza le prestazioni del progetto relativamente ai tre difetti intrinseci del cloud gaming: riduzione della qualità audio-video, bit-rate richiesto e il problema della latenza;
	\item nelle conclusioni si riassumono gli scopi, le valutazioni di questi e le prospettive future;
	\item nell'appendice A si trova il manuale utente.
\end{itemize}