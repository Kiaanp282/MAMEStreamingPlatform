%license:BSD-3-Clause
%copyright-holders:Michele Maione
%============================================================
%
%	Piattaforma di cloud gaming per giochi arcade
%
%============================================================

\chapter*{Direzioni future di ricerca e conclusioni}
\addcontentsline{toc}{chapter}{Direzioni future di ricerca e conclusioni}
\markboth{}{DIREZIONI FUTURE DI RICERCA E CONCLUSIONI}

%Diciamo che di solito a inizio capitolo si "tirano le somme" di quello che è stato fatto e presentato nei capitoli precedenti, riassumendo anche la qualità dei risultati raggiunti.
%Di solito il capitolo è almeno 1 pag e mezzo/2 pagine
%Poi, come hai fatto, si presentano commenti su possibili estensioni/correzioni ecc.
%Non partirei come hai fatto con una considerazione così estemporanea sul gradimento degli utenti, in quanto di solito, nel nostro settore, se fai un test con utenti "serio", allora è necessario un po' più di formalismo (devi dare statistiche sugli utenti, come hai chiesto di provare, testare e valutare, ecc). Visto che non era obiettivo della tua tesi, io eviterei di accennare al fatto che lo hai fatto provare, o al massimo includerei una frase (ma non all'inizio) che dica che da "primi test preliminari con utenti" hai avuto feedback positivo, e che una valutazione con utenti più completa è rimandata a un lavoro futuro.

%Si mostrano le prospettive future di ricerca nell'area dove si è svolto il lavoro. Talvolta questa sezione può essere l'ultima sottosezione della precedente. Nelle conclusioni si deve richiamare l'area, lo scopo della tesi, cosa è stato fatto, come si valuta quello che si è fatto e si enfatizzano le prospettive future per mostrare come andare avanti nell'area di studio.

%Intro di cosa era stato proposto
%-richiamare l'area, lo scopo della tesi

%le caratteristiche principali del progetto, che sono state vincolanti nella scelta delle tecnologie da utilizzare, sono la portabilità e la possibilità di utilizzare il sistema senza dover installare software aggiuntivi; per questi vincoli, lato client, la scelta è ricaduta sul browser web.
In questa tesi è stata proposta una piattaforma di cloud gaming per videogiochi arcade, le cui caratteristiche principali sono state la portabilità e la possibilità di utilizzarla senza dover installare software aggiuntivi. Queste caratteristiche si sono tradotte in una compilazione multipiattaforma lato server e un'interfaccia web lato client.

%Riassunto di come è stato fatto
%-cosa è stato fatto
Il lavoro comprende un'analisi delle varie piattaforme di cloud gaming del passato, di quelle attualmente sul mercato e di alcune piattaforme open source; questa analisi è stata la base di partenza per la progettazione della piattaforma. Il lavoro è continuato con la modifica del codice sorgente del software MAME (licenza GNU-GPL) tramite l'implementazione in C++ delle cinque fasi aggiuntive del cloud gaming: cattura, codifica, decodifica, trasmissione e gestione dell'input utente; utilizzando come formato audiovisivo lo standard \textit{MPEG-1}. La valutazione dei tre difetti intrinseci del cloud gaming e delle prestazioni ottenute concludono il lavoro.

%Performance
%-come si valuta quello che si è fatto
I tre difetti intrinseci del cloud gaming sono stati valutati con l'utilizzo di metodi oggettivi.
La qualità video ha ottenuto un valore di PSNR di $27,703$ \si{dB}, che ha valori tipici tra 20 \si{dB} e 30 \si{dB} per la compressione video per lo streaming. Mentre l'indice SSIM è risultato del $89,9\%$.
La qualità audio ha ottenuto un grado PEAQ di 2 che nella scala ODG corrisponde a fastidioso.
La latenza è stata calcolata cronometrando il round trip delay medio che è risultato di 182 ms; il valore risulta leggermente superiore alla soglia di ritardo per i giochi FPS (100 ms) ma inferiore a quella degli RPG (500 ms).
Il bit-rate richiesto per la risoluzione di 480p oscilla, in base al tipo di gioco, tra $0,5$ e $3,5$ Mbps con una media di $1$ Mbps, rendendo il sistema perfettamente fruibile nelle moderne reti LAN e WAN.

I test preliminari sono stati effettuati con un numero contenuto di utenti dai quali sono stati ricevuti feedback positivi sia riguardo le performance sia riguardo l'interesse che hanno ancora oggi i videogiochi arcade. Una valutazione più completa è rimandata ad un lavoro futuro.

%Sviluppi futuri
%-si enfatizzano le prospettive future per mostrare come andare avanti nell'area di studio
Il progetto apre le porte a molte idee e miglioramenti che possono portarlo a essere effettivamente competitivo. Molto utile per la fidelizzazione dell'utente sarebbe la creazione di un sistema di account tramite cui il giocatore può salvare e caricare lo stato del gioco, pubblicare i punteggi nella leaderboard, invitare altri giocatori a unirsi alla partita supportando così il multiplayer da devices differenti. Per ridurre la dimensione dei pacchetti si potrebbero usare due codec open source che in futuro saranno supportati nativamente dalla maggior parte dei browser: \textit{AOMedia Video 1 (AV1)}, progettato per trasmissioni video su Internet, e \textit{Opus}, un codec audio lossy utilizzato per la comunicazione in tempo reale; entrambi inseribili nel contenitore \textit{WebM}. Per diminuire l'overhead di comunicazione si potrebbe sostituire il protocollo \textit{WebSocket} con \textit{RTP} utilizzabile tramite la tecnologia \textit{WebRTC}. Nella sua versione attuale il progetto è in ascolto su una porta specifica e può essere distribuito su più server utilizzando un software per il bilanciamento del carico (ad esempio \textit{Nginx}), oppure ampliando la classe che si occupa di orchestrare la comunicazione con il client, sfruttando lo stesso protocollo utilizzato per la ricezione dell'input utente si possono scambiare informazioni con altre istanze del MAME CGP in esecuzione su altri server per un efficiente bilanciamento del carico. Un'altra strategia di ottimizzazione, sebbene complessa, consiste nel dividere le varie fasi del cloud gaming in componenti e distribuirle dinamicamente su più server in base alle risorse richieste e disponibili.

%Ok fa quello che doveva fare, e ci si può lavorare sopra
A valle di tutto quello di cui si è parlato finora si ritiene che gli obiettivi iniziali della tesi siano stati raggiunti e che il MAME CGP sia in grado di fungere da piattaforma di cloud gaming. Infine, il fatto che gli sviluppi possibili siano molteplici induce a pensare che il sistema abbia ampi margini evolutivi e che questo lavoro possa stimolare altre persone ad una maggiore innovazione sui sistemi di cloud gaming e sulle tecnologie di streaming in generale.