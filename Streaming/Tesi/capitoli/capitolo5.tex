%license:BSD-3-Clause
%copyright-holders:Michele Maione
%============================================================
%
%	Piattaforma di cloud gaming per giochi arcade
%
%============================================================

\chapter*{Direzioni future di ricerca e conclusioni}
\addcontentsline{toc}{chapter}{Direzioni future di ricerca e conclusioni}

%Si mostrano le prospettive future di ricerca nell'area dove si è svolto il lavoro. Talvolta questa sezione può essere l'ultima sottosezione della precedente. Nelle conclusioni si deve richiamare l'area, lo scopo della tesi, cosa è stato fatto, come si valuta quello che si è fatto e si enfatizzano le prospettive future per mostrare come andare avanti nell'area di studio.

%Intro di cosa era stato proposto
%-richiamare l'area, lo scopo della tesi
In questa tesi è stata proposta una piattaforma di cloud gaming per videogiochi arcade utilizzabile tramite browser da qualsiasi dispositivo collegato alla rete, senza bisogno dell'installazione di software aggiuntivi.

%Riassunto di come è stato fatto
%-cosa è stato fatto
Il lavoro comprende un'analisi delle varie piattaforme di cloud gaming del passato, di quelle attualmente sul mercato e di alcune piattaforme open source; questa analisi è stata la base per la progettazione della piattaforma. Il lavoro è continuato con la modifica del codice sorgente del software MAME (licenza GNU-GPL) tramite l'implementazione in C++ delle cinque fasi aggiuntive del cloud gaming: la cattura, codifica e decodifica audiovisiva, la trasmissione e la gestione dell'input utente; utilizzando come formato audiovisivo lo standard \textit{MPEG-1}. La valutazione dei tre difetti intrinseci del cloud gaming e delle prestazioni ottenute conclude il lavoro.

%Performance
%-come si valuta quello che si è fatto
I primi approcci al cloud gaming risalgono agli anni 2000, ma è solo dieci anni più tardi che venne lanciata sul mercato la piattaforma OnLive di OL2, con una risoluzione video di 480p ed un bit-rate richiesto di 1,5 Mbps. Da allora sono passati altri dieci anni e le attuali tecnologie informatiche hanno reso possibile giocare in cloud con qualità audiovisive pari a quelle delle moderne console, con bit-rate richiesti tra i 10 e i 25 Mbps per la risoluzione a 1080p (Luna, GeForce Now e PlayStation Now) e 35 Mbps per quella a 2160p (Stadia).
I tre difetti intrinseci del cloud gaming sono stati valutati con l'utilizzo di metodi oggettivi:
\begin{itemize}
	\item la qualità video ha ottenuto un valore di PSNR di $27,703$ dB e un indice SSIM del $89,9\%$;
	\item la qualità audio ha ottenuto un grado PEAQ di 2 (fastidioso);
	\item la latenza è stata calcolata cronometrando il round trip delay medio che è risultato di 182 ms;
	\item il bit-rate richiesto per la risoluzione di 480p oscilla, in base al tipo di gioco, tra $0,5$ e $3,5$ Mbps con una media di $1$ Mbps.
\end{itemize}
I test preliminari sono stati effettuati con un numero contenuto di utenti dai quali sono stati ricevuti feedback positivi sia riguardo le performance che riguardo l'interesse che hanno ancora oggi i videogiochi arcade. Una valutazione più completa è rimandata ad un lavoro futuro.

%Sviluppi futuri
%-si enfatizzano le prospettive future per mostrare come andare avanti nell'area di studio
Il progetto infatti apre le porte a molte idee e miglioramenti che possono portarlo a essere effettivamente competitivo. Molto utile per la fidelizzazione dell'utente sarebbe la creazione di un sistema di account tramite cui il giocatore può salvare e caricare lo stato del gioco, pubblicare i punteggi nella leaderboard, invitare altri giocatori a unirsi alla partita supportando così il multiplayer da devices differenti. Per ridurre la dimensione dei pacchetti si potrebbero usare due codec open source che in futuro saranno supportati nativamente dalla maggior parte dei browser: \textit{AOMedia Video 1 (AV1)}, progettato per trasmissioni video su Internet, e \textit{Opus}, un codec audio lossy utilizzato per la comunicazione in tempo reale; entrambi inseribili nel contenitore \textit{WebM}. Per diminuire l'overhead di comunicazione si potrebbe sostituire il protocollo \textit{WebSocket} con \textit{RTP} utilizzabile tramite la tecnologia \textit{WebRTC}. Nella sua versione attuale il progetto è in ascolto su una porta specifica e può essere distribuito su più server utilizzando un software per il bilanciamento del carico (ad esempio \textit{Nginx}), oppure ampliando la classe che si occupa di orchestrare la comunicazione con il client, sfruttando lo stesso protocollo utilizzato per la ricezione dell'input utente si possono scambiare informazioni con altre istanze del MAME CGP in esecuzione su altri server per un efficiente bilanciamento del carico. Un'altra strategia di ottimizzazione, sebbene complessa, consiste nel dividere le varie fasi del cloud gaming in componenti e distribuirle dinamicamente su più server in base alle risorse richieste e disponibili.

%Ok fa quello che doveva fare, e ci si può lavorare sopra
A valle di tutto quello di cui si è parlato finora si ritiene che gli obiettivi iniziali della tesi siano stati raggiunti e che il MAME CGP sia in grado di fungere da piattaforma di cloud gaming. Infine, il fatto che gli sviluppi possibili siano molteplici induce a pensare che il sistema abbia ampi margini evolutivi e che questo lavoro possa stimolare altri sviluppatori ad una maggiore innovazione sui sistemi di cloud gaming e sulle tecnologie di streaming in generale.