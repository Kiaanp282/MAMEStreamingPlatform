%license:BSD-3-Clause
%copyright-holders:Michele Maione
%============================================================
%
%	Piattaforma di cloud gaming per giochi arcade
%
%============================================================

\chapter*{Direzioni future di ricerca e conclusioni}
\addcontentsline{toc}{chapter}{Direzioni future di ricerca e conclusioni}

%Si mostrano le prospettive future di ricerca nell'area dove si è svolto il lavoro. Talvolta questa sezione può essere l'ultima sottosezione della precedente. Nelle conclusioni si deve richiamare l'area, lo scopo della tesi, cosa è stato fatto, come si valuta quello che si è fatto e si enfatizzano le prospettive future per mostrare come andare avanti nell'area di studio.

Durante la realizzazione di questo progetto, molte persone dopo aver verificato il funzionamento hanno chiesto quando sarebbe stato reso pubblico; tale richiesta evidenza il grande interesse che hanno ancora oggi i videogiochi arcade per le persone di tutte le età. Il MAME CGP si prefiggeva lo scopo di rendere fruibili nel modo più semplice possibile e per la maggior parte dei dispositivi i videogiochi arcade, scopo che si è concretizzato sottoforma di piattaforma di cloud gaming. Credo fermamente che il rilascio del codice sorgente del MAME CGP stimolerà altri sviluppatori ad una maggiore innovazione sui sistemi di cloud gaming e sulle tecnologie di streaming in generale, nonché ad ampliamenti del progetto.

Il progetto apre le porte a molte idee e miglioramenti che possono portarlo ad essere effettivamente competitivo. Molto utile per la fidelizzazione dell'utente sarebbe la creazione di un sistema di account tramite cui il giocatore può salvare e caricare lo stato del gioco, pubblicare i punteggi nella leaderboard, invitare altri giocatori ad unirsi alla partita supportando così il multiplayer da devices differenti. Per ridurre la dimensione dei pacchetti si potrebbero usare due codec open-source che in futuro saranno supportati nativamente dalla maggior parte dei browser: \textit{AOMedia Video 1 (AV1)}, progettato per trasmissioni video su Internet, e \textit{Opus}, un codec audio lossy utilizzato per la comunicazione in tempo reale. Entrambi i codec sono inseribili nel contenitore \textit{WebM}. Per diminuire l'overhead di comunicazione si potrebbe sostituire il protocollo \textit{WebSocket} con \textit{RTP} utilizzabile tramite la tecnologia \textit{WebRTC}. Nella sua versione attuale il progetto è in ascolto su una porta specifica e può essere distribuito su più server utilizzando un software per il bilanciamento del carico (ad esempio \textit{Nginx}), oppure ampliando la classe che si occupa di orchestrare la comunicazione con il client, sfruttando lo stesso protocollo utilizzato per la ricezione dell'input utente si possono scambiare informazioni con altre istanze del MAME CGP in esecuzione su altri server per un'efficiente bilanciamento del carico. Un'altra strategia di ottimizzazione, sebbene complessa, consiste nel dividere le varie fasi del cloud gaming in componenti e distribuirle dinamicamente su più server in base alle risorse richieste e disponibili.

A valle di tutto quello di cui si è parlato finora si ritiene che gli obiettivi iniziali della tesi siano stati raggiunti e che il MAME CGP è in grado di fungere da piattaforma di cloud gaming. Infine, il fatto che gli sviluppi possibili siano molteplici induce a pensare che il sistema abbia ampi margini evolutivi.