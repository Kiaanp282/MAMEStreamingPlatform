%license:BSD-3-Clause
%copyright-holders:Michele Maione
%============================================================
%
%	Piattaforma di cloud gaming per giochi arcade
%
%============================================================

\chapter*{Direzioni future di ricerca e conclusioni}
\addcontentsline{toc}{chapter}{Direzioni future di ricerca e conclusioni}

%Si mostrano le prospettive future di ricerca nell'area dove si è svolto il lavoro. Talvolta questa sezione può essere l'ultima sottosezione della precedente. Nelle conclusioni si deve richiamare l'area, lo scopo della tesi, cosa è stato fatto, come si valuta quello che si è fatto e si enfatizzano le prospettive future per mostrare come andare avanti nell'area di studio.

Durante la realizzazione di questo progetto molte persone che l'hanno provato mi hanno chiesto quando l'avrei reso pubblico per poterlo utilizzare, ed è evidente il grande interesse che hanno ancora oggi i videogiochi arcade per le persone di tutte le età. Il MAME CGP si prefiggeva lo scopo di rendere fruibili nel modo più semplice possibile e per la maggior parte dei dispositivi i videogiochi arcade, che si è concretizzato sottoforma di piattaforma di cloud gaming. Credo fermamente che il rilascio di MAME CGP stimolerà altri sviluppatori ad una maggiore innovazione sui sistemi di cloud gaming e sulle tecnologie di streaming in generale, nonché ad ampliamenti del progetto. 

Molto utile per il suo utilizzo su internet sarebbe la creazione di un sistema di account tramite cui il giocatore potrebbe salvare e caricare lo stato del gioco, pubblicare i punteggi nella leaderboard, invitare altri giocatori ad unirsi alla partita supportando così il multiplayer da devices differenti.

Per ridurre la dimensione dei pacchetti si potrebbero usare due codec open-source che in futuro saranno supportati nativamente dalla maggior parte dei browser: AOMedia Video 1 (AV1), progettato per trasmissioni video su Internet, e Opus, un codec audio lossy utilizzato per la comunicazione in tempo reale. Ambedue inseribili nel contenitore WebM.

Per diminuire l'overhead di comunicazione si potrebbe sostituire il protocollo WebSocket con RTP utilizzabile tramite la tecnologia WebRTC.

Nella sua versione attuale il progetto ascolta su una porta specifica e può essere distribuito su più server utilizzando un software per il bilanciamento del carico. Un'altra strategia di ottimizzazione, sebbene complessa, consiste nel dividere le varie fasi del cloud gaming in componenti e distribuirle dinamicamente su più server in base alle risorse richieste e disponibili.

Si ritiene, quindi, che gli obiettivi iniziali della tesi siano stati raggiunti, ottenendo un sistema in grado di produrre per le notizie i punteggi di fakeness desiderati. Infine, il fatto che gli sviluppi possibili sono molteplici induce a pensare che, potenzialmente, il sistema proposto abbia ampi margini di miglioramento.