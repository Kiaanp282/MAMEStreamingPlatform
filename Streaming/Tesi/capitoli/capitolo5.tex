%license:BSD-3-Clause
%copyright-holders:Michele Maione
%============================================================
%
%	Piattaforma di cloud gaming per giochi arcade
%
%============================================================

\chapter*{Direzioni future di ricerca e conclusioni}
\addcontentsline{toc}{chapter}{Direzioni future di ricerca e conclusioni}

%Si mostrano le prospettive future di ricerca nell'area dove si è svolto il lavoro. Talvolta questa sezione può essere l'ultima sottosezione della precedente. Nelle conclusioni si deve richiamare l'area, lo scopo della tesi, cosa è stato fatto,come si valuta quello che si è fatto e si enfatizzano le prospettive future per mostrare come andare avanti nell'area di studio.
Come detto precedentemente il progetto è stato pensato per essere utilizzato in uno stand per il retrogaming, ma se utilizzato come servizio web sarebbe necessario creare un sistema di account tramite cui il giocatore potrebbe salvare e caricare lo stato del gioco, pubblicare i punteggi nella leaderboard, invitare altri giocatori ad unirsi alla partita supportando così il multiplayer da devices differenti.

Per ridurre la dimensione dei pacchetti si potrebbero usare due codec open-source che in futuro saranno supportati nativamente dalla maggior parte dei browser: AOMedia Video 1 (AV1), progettato per trasmissioni video su Internet, e Opus, un codec audio lossy utilizzato per la comunicazione in tempo reale. Ambedue inseribili nel contenitore WebM.

Per diminuire l'overhead di comunicazione si potrebbe sostituire il protocollo WebSocket con RTP utilizzabile tramite la tecnologia WebRTC.
