%license:BSD-3-Clause
%copyright-holders:Michele Maione
%============================================================
%
%	Piattaforma di cloud gaming per giochi arcade
%
%============================================================
\documentclass{article} %pdf

\usepackage[T1]{fontenc}
\usepackage[a4paper]{geometry}
%\usepackage[main=italian]{babel} %stesura
\usepackage[main=italian, english]{babel} %prod
\usepackage[utf8]{inputenc}
\usepackage[autostyle,italian=guillemets]{csquotes}
\usepackage[a-1b]{pdfx}
\usepackage{float}
\usepackage{emptypage}
\usepackage{tablefootnote}
\usepackage{graphicx}
\usepackage{amssymb,amsmath,amsthm}
\usepackage{mathtools}
\usepackage{algorithm,algpseudocode} 
\usepackage{verbatim}
\usepackage{commath}
\usepackage{caption}
\usepackage{url}
\usepackage{siunitx}
\usepackage[style=ext-authoryear,sorting=nty,maxbibnames=15]{biblatex}

\DeclareOuterCiteDelims{parencite}{\bibopenbracket}{\bibclosebracket}

\definecolor{grigiochiaro}{RGB}{245,245,245}
\definecolor{grigio_commento}{RGB}{120,120,120}
\definecolor{grigio}{RGB}{100,100,100}

\captionsetup[figure]{font={footnotesize, color=grigio}}
\captionsetup[lstlisting]{font={footnotesize, color=grigio}}
\captionsetup[table]{font={footnotesize, color=grigio}}

\hypersetup{hidelinks}

\addbibresource{Bibliografia.bib}

\title{Piattaforma di cloud gaming per giochi arcade}
\author{Michele Maione - 931468}
\date{Milano, ottobre 2021}


\begin{document}
\maketitle
\addcontentsline{toc}{chapter}{Piattaforma di cloud gaming per giochi arcade}

Negli ultimi anni sono apparse molte piattaforme che sfruttano il paradigma del cloud computing per offrire servizi accessibili su richiesta e da remoto per archiviare file, utilizzare le suite per l'ufficio, vedere film e serie TV, ascoltare musica e, a partire dal 2011, anche giocare.

Il cloud gaming è un servizio che unisce il cloud computing e il live streaming per rendere possibile giocare in remoto senza scaricare o installare il gioco sul device dell'utente, in pratica consente di archiviare ed eseguire i videogiochi su un server remoto e trasmettere l'output audio-video all'utente sul proprio dispositivo.

Lo scopo di questa tesi è creare una piattaforma di cloud gaming per far conoscere alle nuove generazioni i videogiochi che hanno fatto la storia e dare la possibilità di poter giocare ancora a macchine che ormai hanno cessato di funzionare per motivi di obsolescenza. Per far ciò verrà ampliato il software MAME (rilasciato sotto licenza GNU-GPL) che è in grado di emulare oltre $7.000$ giochi arcade, in modo che possa fungere da server di cloud gaming e comunicare con un front-end HTML, rimanendo sempre indipendente dal sistema operativo, rendendo più agevole l'installazione di uno stand per il retrogaming. 

Per la realizzazione del progetto sono state prese in considerazione le analisi fatte sulle piattaforme delle multinazionali come GeForce Now, Stadia e PlayStation Now in \parencite{A_Network_Analysis_on_Cloud_Gaming_Stadia_GeForce_Now_and_PSNow} e \parencite{Network_Analysis_of_the_Sony_Remote_Play_System}, come Amazon Luna in \parencite{Amazon_Luna_WebRTC}. Ma anche progetti open source come "Games on Demand" di \parencite{ARealTimeStreamingGamesonDemandSystem} che propone una piattaforma basata sul hooking di funzioni DirectX, codifica in MPEG-2 e trasmissione tramite UDP; e "GamingAnywhere" di \parencite{GamingAnywhere} che è un progetto multipiattaforma che trasmette tramite protocollo RTP ed esegue la cattura audio-video utilizzando la libreria SDL tramite polling.

Il sistema proposto è stato progettato con un'ottica incentrata sull'utilizzo in LAN con l'utenza connessa tramite WiFi, ad esempio in stand di retrogaming ad eventi di informatica e videogiochi, in aziende come servizio di svago per i clienti in sala d'attesa e per i dipendenti durante la pausa, ecc\dots. È importante ricordare che i videogiochi, nonostante siano stati pensati come fonte d'intrattenimento, migliorano diversi tipi di abilità chiave: abilità sociali e intellettuali, riflessi e concentrazione \parencite{Use_of_Cloud_Gaming_in_Education}; per questo motivo la piattaforma può essere installata anche nelle scuole.

Per ampliare il progetto MAME, lato server ho modificato le funzionalità di rendering video e mixing audio per convogliare il loro output, che viene codificato in MPEG-TS, ad un modulo che esegue lo streaming tramite il protocollo WebSocket ad una pagina HTML. Lato client ho creato un modulo JavaScript per gestire l'input utente e decodificare il filmato MPEG-TS. La piattaforma utilizza un bit-rate tra $0,5$ Mbps e $3,5$ Mbps ed offre una risoluzione di 480p; per valutarne la latenza è stata testata su rete locale e su rete internet \parencite{Latency_analysis_for_M2M}; mentre la qualità audio-video è stata classificata tramite "peek signal-to-noise ratio", "structural similarity index method" e "perceptual evaluation of audio quality" \parencite{Cloud_Gaming_Architecture_and_Performance}.

Dai risultati ottenuti si ritiene che gli obiettivi iniziali della tesi siano stati raggiunti e che il progetto è in grado di fungere da piattaforma di cloud gaming. Infine, il fatto che gli sviluppi possibili siano molteplici induce a pensare che il sistema abbia ampi margini evolutivi e che questo lavoro possa stimolare altre persone ad una maggiore innovazione sui sistemi di cloud gaming e sulle tecnologie di streaming in generale.

\printbibliography[nottype=misc,title={Bibliografia},heading=bibintoc]
\printbibliography[type=misc,title={Sitografia},heading=bibintoc]

\end{document}