%license:BSD-3-Clause
%copyright-holders:Michele Maione
%============================================================
%
%	Piattaforma di cloud gaming per giochi arcade
%
%============================================================

\chapter*{Introduzione}
Per far conoscere alle nuove generazioni i videogiochi che hanno fatto la storia e dare la possibilità di poter giocare ancora a macchine che ormai hanno cessato di funzionare per motivi di obsolescenza, sfruttando due tecnologie entrate a far parte della quotidianità, il live streaming e il cloud computing, in questo lavoro si propone la creazione di una piattaforma di cloud gaming, che permette lo streaming audio-video direttamente e su richiesta dei videogiochi, da un server remoto, ad un client (computer, console, telefono). Il gioco è archiviato, eseguito, e renderizzato su un server remoto; l'input (tastiera, gamepad) viene inviato dal client al server e lì processato. In questo modo si può accedere ai giochi indipendentemente dal sistema operativo e dalle capacità hardware del client utilizzato. Inoltre il cloud gaming permette di iniziare a giocare immediatamente poiché il gioco è già installato sul server offrendo agli utenti un rapido accesso. Infine la piattaforma, indirettamente, garantisce la gestione dei diritti digitali (DRM) per gli editori. Per questo progetto verrà ampliato il software MAME (rilasciato sotto licenza GNU-GPL) che è in grado di emulare oltre 7.000 giochi arcade, di modo che possa fungere da server di cloud gaming e comunicare con un front-end HTML, rimanendo sempre indipendente dal sistema operativo, così da rendere più agevole l’installazione di uno stand per il retro-gaming.

Nel Capitolo 1 viene introdotto il cloud gaming e fatta una panoramica dei servizi presenti sul mercato; nel Capitolo 2 viene presentata l'architettura del sistema, mentre si descrivono le tecnologie utilizzate e le funzionalità create nel Capitolo 3. All’interno del Capitolo 4 vengono illustrati i risultati ottenuti. Nell'ultimo capitolo sono riportate le conclusioni finali e una lista di possibili sviluppi futuri.