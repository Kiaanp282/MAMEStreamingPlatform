\chapter{Introduction}
\label{cap:introduction}
Cloud gaming, is a type of online service that operate in a similar way to remote desktop and video on demand. Video games are stored and run remotely on a provider's dedicated hardware, they are streamed as a movie to a player's device via client. The client manages the player's inputs, which are sent to the server and executed in the game.

This type of approach offers many advantages, including making the game more easily accessible without having to download and install it locally, it's compatible with computers and smart phones, even on smart tv when used with a wifi gamepad. Several services may offer some additional functions to make the most of this model, a spectator can join a player's session and temporarily take control of the game, if authorized by the player himself.

It also definitively solves a problem that has existed since the days of compact cassettes and floppy disks: piracy.

The rapid development of the broadband networks and the continuously falling subscription costs made this method, today, a reality.
\section{Existing cloud gaming services}
The first hints of cloud gaming for the general public only came around 2010. One of the first platforms created to allow gamers from all over the world (or almost) to experience the thrill of streaming gaming was Onlive. Presented at GDC 2009 (Game Developer Conference), then launched on the market in June 2010. What made this service different from Steam or other digital stores of the time? It simply allowed access to a cloud computing infrastructure on which the titles owned by the user in the Steam library or purchased directly in the platform store were run. Many titles arrived on OnLive, and many publishers decided to believe in the project, among them we find Ubisoft, THQ and an Epic Games very different from how we see it today.

At the time, all titles ran in basic SD resolutions, and a simple 5Mbps connection was enough to achieve HD. Needless to say, the technology of the time was very different from how we know it today, the same infrastructure of OnLive did not allow the player to have a low latency to be able to play in peace, especially in the most frenetic titles. What happened to OnLive then? Well, it closed its doors around mid-2015 to then be "acquired" by Sony, together with another cloud gaming service (GAIKAI) in order to lay the foundations of the infrastructure of its own streaming service, which today we know as Playstation Now.

We actually know the rest of the story, because most of the services that were born after Onlive, are still alive beyond a few small cases, such as Project Atlas by EA, which then ended up in oblivion, perhaps due to little interest of the same company in making develop the service. In fact, it should be remembered that the case of Onlive must be a warning for all the realities that came later, it was not easy for a young company to focus on something completely new, especially in the field of gaming.

Allowing players from all over the world to play their favorite titles without the aid of hardware is not easy. At the base there must be a well-defined and above all well-distributed infrastructure throughout the territory. Today, however, we have many cloud gaming services that allow us to access different entertainment offers, according to our needs.
\subsection{Utomik}
The Utomik platform launched commercially in 2008 and has been in service since. Games to be played in a browser need the proprietary Utomik Player plugin. The platform offer an SDK, plugin and online service to create, launch, maintain and monitor games\cite{Utomik}.
\subsection{Microsoft - Xbox Cloud Gaming}
Microsoft teased the service at E3 2018. The service is available for subscribers of Xbox Game Pass Ultimate from September 15, 2020. The platform offer the existing library of Xbox games and add new games from the Xbox Series X. The service is designed to work with phones (currently only Android), either with touchscreen controls or Xbox controller over Bluetooth.\cite{Xbox_Game_Pass_cloud_gaming}.
\subsection{Project Atlas}
In May 2018 Electronic Arts has unveiled its cloud-native gaming platform called Project Atlas, which aims to make numerous playable titles available through the company's servers and to provide a gaming experience never tried before thanks to the support of artificial intelligence. The platform intends to offer a cloud gaming experience that consists of truly living universes, which change with the passage of time, with interaction with other players and under the influence of the outside world. In these processes, the support of artificial intelligence and the learning of the habits and preferences of the players would play a fundamental role. It also offers a dynamic game client, which allows users to stream a title while they wait for the download to complete on their device's storage disk.\cite{Project_Atlas}.
\subsection{Nvidia - GeForce Now}
AA sajk sjskaj ksaj ksa. AA sajk sjskaj ksaj ksa. AA sajk sjskaj ksaj ksa. AA sajk sjskaj ksaj ksa. AA sajk sjskaj ksaj ksa. AA sajk sjskaj ksaj ksa. AA sajk sjskaj ksaj ksa. AA sajk sjskaj ksaj ksa. AA sajk sjskaj ksaj ksa. AA sajk sjskaj ksaj ksa. AA sajk sjskaj ksaj ksa. AA sajk sjskaj ksaj ksa. AA sajk sjskaj ksaj ksa. AA sajk sjskaj ksaj ksa. AA sajk sjskaj ksaj ksa. AA sajk sjskaj ksaj ksa. AA sajk sjskaj ksaj ksa. AA sajk sjskaj ksaj ksa. AA sajk sjskaj ksaj ksa. AA sajk sjskaj ksaj ksa\cite{GeForce_Now}.
\subsection{Sony - PlayStation Now}
AA sajk sjskaj ksaj ksa. AA sajk sjskaj ksaj ksa. AA sajk sjskaj ksaj ksa. AA sajk sjskaj ksaj ksa. AA sajk sjskaj ksaj ksa. AA sajk sjskaj ksaj ksa. AA sajk sjskaj ksaj ksa. AA sajk sjskaj ksaj ksa. AA sajk sjskaj ksaj ksa. AA sajk sjskaj ksaj ksa. AA sajk sjskaj ksaj ksa. AA sajk sjskaj ksaj ksa. AA sajk sjskaj ksaj ksa. AA sajk sjskaj ksaj ksa. AA sajk sjskaj ksaj ksa. AA sajk sjskaj ksaj ksa. AA sajk sjskaj ksaj ksa. AA sajk sjskaj ksaj ksa. AA sajk sjskaj ksaj ksa. AA sajk sjskaj ksaj ksa\cite{PlayStation_Now}.
\subsection{Google - Stadia}
AA sajk sjskaj ksaj ksa. AA sajk sjskaj ksaj ksa. AA sajk sjskaj ksaj ksa. AA sajk sjskaj ksaj ksa. AA sajk sjskaj ksaj ksa. AA sajk sjskaj ksaj ksa. AA sajk sjskaj ksaj ksa. AA sajk sjskaj ksaj ksa. AA sajk sjskaj ksaj ksa. AA sajk sjskaj ksaj ksa. AA sajk sjskaj ksaj ksa. AA sajk sjskaj ksaj ksa. AA sajk sjskaj ksaj ksa. AA sajk sjskaj ksaj ksa. AA sajk sjskaj ksaj ksa. AA sajk sjskaj ksaj ksa. AA sajk sjskaj ksaj ksa. AA sajk sjskaj ksaj ksa. AA sajk sjskaj ksaj ksa. AA sajk sjskaj ksaj ksa\cite{Google_Stadia}.
\subsection{Amazon - Luna}
Luna was announced on September 24, 2020, with ‘early access’ available to subscribers by invitation beginning on October 20, 2020. Amazon Luna will have 100 different games at launch and will be powered by AWS. Luna will have integration with Twitch. Amazon has partnered with Ubisoft to create a gaming channel (with an additional costs) exclusive to Luna, which will give Luna subscribers access to Ubisoft's titles the same day they release\cite{Amazon_Luna}.