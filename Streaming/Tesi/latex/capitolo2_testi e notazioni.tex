\chapter{Testi, citazioni e formule}
\label{chap:testi}
\section{Testi}
Questa è la prima sezione del capitolo.

Andando a capo due volte si crea un nuovo paragrafo.
\\
Con doppio backslash si evita il rientro.
\\
Così si citano le cose della bibliografia:

\cite{wikipedia}
\cite{CPP_Primer}
\cite{Computer_Networking_and_the_Internet}
\cite{Ingegneria_del_software}
\cite{Understanding_the_Linux_Kernel}
\cite{Windows_Server_2012}

\subsection{Sottosezione}
sono una sottosezione
\subsubsection{sotto sotto sezione}
sono una piccola sottosezione

\section{Corsivo}
Sezione due. \textit{textit per fare il corsivo} 

\section{Formule}

\[p(x) = a^2 - g_{123}\]
\[a = \frac{ciao}{come va?}\]


Piccola formula: $a=b(x)$

altra equazione:

\begin{equation}
E=m
\end{equation}

Mettere no number per non numerarla.
\begin{equation}
E=m \nonumber
\end{equation}

