%license:BSD-3-Clause
%copyright-holders:Michele Maione
%============================================================
%
%	Piattaforma di cloud gaming per giochi arcade
%
%============================================================

\chapter{Conclusioni e sviluppi futuri}

Come detto precedentemente il progetto è stato pensato per essere utilizzato in uno stand per il retrogaming, ma se utilizzato come servizio web sarebbe necessario creare un sistema di account tramite cui il giocatore potrebbe salvare e caricare lo stato del gioco, pubblicare i punteggi nella leaderboard, invitare altri giocatori ad unirsi alla partita supportando così il multiplayer da devices differenti.

Per ridurre la dimensione dei pacchetti si potrebbero usare due codec open-source che in futuro saranno supportati nativamente dalla maggior parte dei browser: AOMedia Video 1 (AV1), progettato per trasmissioni video su Internet, e Opus, un codec audio lossy utilizzato per la comunicazione in tempo reale. Ambedue inseribili nel contenitore WebM.

Per diminuire l'overhead di comunicazione si potrebbe sostituire il protocollo WebSocket con RTP.