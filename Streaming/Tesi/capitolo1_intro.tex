\chapter{Introduction}
\label{cap:introduction}
Cloud gaming, is a type of online service that operate in a similar way to remote desktop and video on demand. Video games are stored and run remotely on a provider's dedicated hardware, they are streamed as a movie to a player's device via client. The client manages the player's inputs, which are sent to the server and executed in the game.

This type of approach offers many advantages, including making the game more easily accessible without having to download and install it locally, it's compatible with computers and smart phones, even on smart tv when used with a wifi gamepad. Several services may offer some additional functions to make the most of this model, a spectator can join a player's session and temporarily take control of the game, if authorized by the player himself.

It also definitively solves a problem that has existed since the days of compact cassettes and floppy disks: piracy.

The rapid development of the broadband networks and the continuously falling subscription costs made this method, today, a reality.

\section{Existing cloud gaming services}
The first hints of cloud gaming for the general public only came around 2010. One of the first platforms created to allow gamers from all over the world to experience the thrill of streaming gaming was OnLive by OL2. Presented at GDC 2009, then launched on the market in June 2010. Players could purchase games on the platform or play games from their Steam library.

In 2012, Gaikai inaugurated its cloud gaming service, the company mainly focused on using cloud gaming as a form of online advertising for video games, where users would have the opportunity to access video game demos.

OnLive and Gaikai was acquired by Sony Interactive Entertainment and their resources were used as the basis for a cloud gaming service known as Sony Playstation Now.

In 2013, Nvidia introduced GeForce Now, a cloud gaming service integrated in their SHIELD Android TV device. In 2017, the company began expanding its service to PC, including support for importing a user's Steam library.

In May 2018, Electronic Arts acquired some cloud gaming assets from GameFly. EA later announced Project Atlas, a project that explored the integration of artificial intelligence and machine learning, making the platform dynamic, social and cross platform.

Microsoft at E3 2018 teased his Xbox Cloud Gaming service.

At GDC 2019, Google officially announced its Stadia cloud gaming service, out for November 19 of the same year.

On September 24, 2020, a new service called Amazon Luna was announced by Amazon\cite{Cloud_gaming_history}.
\subsection{Utomik}
The Utomik platform launched commercially in 2008 and has been in service since. Games to be played in a browser need the proprietary Utomik Player plugin. The platform offer an SDK, plugin and online service to create, launch, maintain and monitor games\cite{Utomik}.
\subsection{Microsoft - Xbox Cloud Gaming}
Microsoft teased the service at E3 2018. The service is available for subscribers of Xbox Game Pass Ultimate from September 15, 2020. The platform offer the existing library of Xbox games and add new games from the Xbox Series X. The service is designed to work with phones (currently only Android), either with touchscreen controls or Xbox controller over Bluetooth.\cite{Xbox_Game_Pass_cloud_gaming}.
\subsection{Project Atlas}
In May 2018 Electronic Arts has unveiled its cloud-native gaming platform called Project Atlas, which aims to make numerous playable titles available through the company's servers and to provide a gaming experience never tried before thanks to the support of artificial intelligence. The platform intends to offer a cloud gaming experience that consists of truly living universes, which change with the passage of time, with interaction with other players and under the influence of the outside world. In these processes, the support of artificial intelligence and the learning of the habits and preferences of the players would play a fundamental role. It also offers a dynamic game client, which allows users to stream a title while they wait for the download to complete on their device's storage disk.\cite{Project_Atlas}.
\subsection{Nvidia - GeForce Now}
AA sajk sjskaj ksaj ksa. AA sajk sjskaj ksaj ksa. AA sajk sjskaj ksaj ksa. AA sajk sjskaj ksaj ksa. AA sajk sjskaj ksaj ksa. AA sajk sjskaj ksaj ksa. AA sajk sjskaj ksaj ksa. AA sajk sjskaj ksaj ksa. AA sajk sjskaj ksaj ksa. AA sajk sjskaj ksaj ksa. AA sajk sjskaj ksaj ksa. AA sajk sjskaj ksaj ksa. AA sajk sjskaj ksaj ksa. AA sajk sjskaj ksaj ksa. AA sajk sjskaj ksaj ksa. AA sajk sjskaj ksaj ksa. AA sajk sjskaj ksaj ksa. AA sajk sjskaj ksaj ksa. AA sajk sjskaj ksaj ksa. AA sajk sjskaj ksaj ksa\cite{GeForce_Now}.
\subsection{Sony - PlayStation Now}
AA sajk sjskaj ksaj ksa. AA sajk sjskaj ksaj ksa. AA sajk sjskaj ksaj ksa. AA sajk sjskaj ksaj ksa. AA sajk sjskaj ksaj ksa. AA sajk sjskaj ksaj ksa. AA sajk sjskaj ksaj ksa. AA sajk sjskaj ksaj ksa. AA sajk sjskaj ksaj ksa. AA sajk sjskaj ksaj ksa. AA sajk sjskaj ksaj ksa. AA sajk sjskaj ksaj ksa. AA sajk sjskaj ksaj ksa. AA sajk sjskaj ksaj ksa. AA sajk sjskaj ksaj ksa. AA sajk sjskaj ksaj ksa. AA sajk sjskaj ksaj ksa. AA sajk sjskaj ksaj ksa. AA sajk sjskaj ksaj ksa. AA sajk sjskaj ksaj ksa\cite{PlayStation_Now}.
\subsection{Google - Stadia}
AA sajk sjskaj ksaj ksa. AA sajk sjskaj ksaj ksa. AA sajk sjskaj ksaj ksa. AA sajk sjskaj ksaj ksa. AA sajk sjskaj ksaj ksa. AA sajk sjskaj ksaj ksa. AA sajk sjskaj ksaj ksa. AA sajk sjskaj ksaj ksa. AA sajk sjskaj ksaj ksa. AA sajk sjskaj ksaj ksa. AA sajk sjskaj ksaj ksa. AA sajk sjskaj ksaj ksa. AA sajk sjskaj ksaj ksa. AA sajk sjskaj ksaj ksa. AA sajk sjskaj ksaj ksa. AA sajk sjskaj ksaj ksa. AA sajk sjskaj ksaj ksa. AA sajk sjskaj ksaj ksa. AA sajk sjskaj ksaj ksa. AA sajk sjskaj ksaj ksa\cite{Google_Stadia}.
\subsection{Amazon - Luna}
Luna was announced on September 24, 2020, with ‘early access’ available to subscribers by invitation beginning on October 20, 2020. Amazon Luna will have 100 different games at launch and will be powered by AWS. Luna will have integration with Twitch. Amazon has partnered with Ubisoft to create a gaming channel (with an additional costs) exclusive to Luna, which will give Luna subscribers access to Ubisoft's titles the same day they release\cite{Amazon_Luna}.